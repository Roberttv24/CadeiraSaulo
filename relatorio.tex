\documentclass[a4paper, 12pt]{article}

% Pacotes essenciais
\usepackage[utf8]{inputenc}    % Suporte a caracteres UTF-8
\usepackage[brazil]{babel}     % Idioma em português brasileiro
\usepackage{amsmath, amssymb}  % Pacotes matemáticos
\usepackage{graphicx}          % Inclusão de imagens
\usepackage{geometry}          % Ajuste de margens
\geometry{a4paper, margin=2.5cm}

% Personalização de cabeçalho e rodapé
\usepackage{fancyhdr}
\pagestyle{fancy}
\fancyhf{}
\fancyhead[L]{Tópicos de Física Teórica I}
\fancyhead[R]{\thepage}
\fancyfoot[C]{\small Universidade Federal de Fortaleza - Departamento de Física}

% Título e autorssaes
\title{\textbf{1° Lista}}
\author{
    Robert Bertoldo Tavares \\
    Aluno 2 \\
    Aluno 3
}
\date{\today}

\begin{document}

\maketitle
\thispagestyle{fancy}

\section*{Problema 1}
\section*{Problema 2}
\section*{Problema 3}
Usando o importance sampling, conseguimos calcular o valor da integral. A amostragem se mostrou 
muito relevante para a exatidão do valor calculado. Para amostras entre \(1000\) e \(100000\), o erro
foi acima de \(1\%\). Para uma amostragem de \(1000000\) ou superior, o erro foi sempre inferior a \(1\%\),
mas o custo computacional tambem aumentou siginificativamente, indo de decimos para dezenas de segundos. 
Um dos melhores resultados que obtivemos foi:
\\

\noindent\textbf{Monte Carlo: 1.08049}\\
\textbf{Teórico: 1.08232}\\
\textbf{Erro: 0.1695}


\end{document}
